\chapter{Conclusion}
This research was done in order to create a process used to retrieve the participation ratios of lossy layers in the storing of energy in the electric field around a transmon qubit of arbitrary shape. The lossy layers created during the production of the qubit are a prominent source of decoherence of the qubit. By lowering the energy stored in these layers their influence on the decoherence may be reduced. A simplified model of the system was created in order to simulate in the 3D EM simulation software 'CST'. The retrieved electric field was used to calculate the energy stored in the lossy layers. The total energy stored by the qubit at resonance was calculated by treating the system as an LC-circuit where the capacitances of the qubit are represented by a single equivalent capacitor and the Josephson junction is replaced by an inductor. The participation ratios of the lossy layers was then calculated. 
A second goal was to determine the affect of changing the layout of the qubit´s capacitor pads. The starting point was the interdigitated qubit. By separately changing parameters of this layout their influence on the participation ratio was determined. 
The largest reduction in participation ratios was achieved by elimination of the 'fingers' of the interdigitated qubit entirely. The result is the parallel pad qubit. It has smaller participation ratios for all lossy layers at equal capacitance.  Further reduction can be achieved by increasing the separation of the pads. Both adjustments to the qubit design decrease the capacitance of the system. To compensate, the overall size of the capacitor pads must be increased. As space is a precious commodity in the design of these qubits, this could be a mayor drawback.

During this research the available system memory put a limit on the refinement of the mesh used by CST. In the future this limit should be increased to allow for more accurate simulations.

In future qubit design, the capacitor pads should have a shape such that the electric field is spread out equally across as large an area as possible. The biggest constraint is the overall size of the resulting qubit system. With this in mind, the results in this research show that the rounding of corners is a promising adjustment to the qubit design. 