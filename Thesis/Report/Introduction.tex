\chapter{Introduction}
Since the introduction of the transmon quantum bit (transmon qubit) by Koch et al. in 2007 \cite{Koch2007} as a promising candidate of qubits there have been investigations into sources of decoherence of these qubits. C. Wang et al. found that surface dielectric dissipation is probably still the major limiting factor for the coherence time of transmon qubits \cite{Wang2015}. The different surface dielectrics introduced to the system during production have distinct material compositions \cite{Bruno2015} and as a result will have a different impact on the coherence time \cite{Wang2015}. Qubit structure design itself will dictate how the Electric field is distributed through the dielectrics. \\The goal of this research is to determine this distribution and to use this information to design a transmon qubit in such a way as to be able to avoid concentrating the Electric field in regions containing more lossy dielectric material. Being able to do so may better the ability to design transmon qubits with longer coherence times. \\ The following section will provide necessary background information to substantiate the above. Information particularly relevant to this research will be provided in the next chapter.

\section{Quantum computing and quantum bits}
\begin{quote}
	-General information about quantum computing. Benefits, application etc.
	-quantum bits; importance of longer coherence time
	
	Restatement of the problem
	-Role of dielectric lossy materials
	-why is this research important?!
	-Knowing how design choices influence the participation ratio of lossy layers.
	
	Restatement of the response
	- “In order to address this problem, I will …”.
	
	Roadmap
	-How will the thesis proceed
\end{quote}


%1. Context: What your audience will need to know in order to understand the problem you are going to confront. This background material will be familiar rather than novel to your target audience; it may act as a refresher or even a primer, but will not cover new ground. I usually suggest that students try to form a template sentence that they can then use as a prompt to help them sketch out each of the three moves. For instance, “Over the past two decades, research in this field has focused on … ”.
%
%2. Problem (and Significance): What isn’t yet well understood. That is, the problem statement will explain what you want to understand (or reveal or explain or explore or reinterpret or contest) and why it will matter to have done so. For instance, “However, [topic] is still poorly understood (or under-examined or excluded or misinterpreted). This lack of attention is significant because knowing [about this topic] will provide a benefit OR not knowing [about this topic] will incur a cost”.
%
%Given the importance of establishing significance and given the frequency with which this step is neglected, I have often wondered about framing it as a separate step. I haven’t done so, for two reasons. First, the three moves are so well established; it seems needlessly confusing to disrupt that familiarity by talking about four moves. Second, and more important, the problem and significance are genuinely connected; it doesn’t make sense to treat the problem and significance separately, even if doing so would encourage us to pay more attention to the significance. The significance is requisite for the problem, not separate from it.
%
%3. Response: What you are actually going to do in your research. For instance, “In order to address this problem, I will …”.
%
%The beauty of this basic model is, of course, is that it makes a great deal of intuitive sense. When students hear it for the first time, they generally feel an immediate sense of familiarity. That intuition doesn’t, however, necessarily make it easy for them to deploy it in their own writing. I focus on four things about this model that may help writers deepen their understanding and thus be better able to use these moves proficiently.
%
%
%
%
%Introduction to the introduction: The first step will be a short version of the three moves, often in as little as three paragraphs, ending with some sort of transition to the next section where the full context will be provided.
%
%Context: Here the writer can give the full context in a way that flows from what has been said in the opening. The extent of the context given here will depend on what follows the introduction; if there will be a full lit review or a full context chapter to come, the detail provided here will, of course, be less extensive. If, on the other hand, the next step after the introduction will be a discussion of method, the work of contextualizing will have to be completed in its entirely here.
%
%Restatement of the problem: With this more fulsome treatment of context in mind, the reader is ready to hear a restatement of the problem and significance; this statement will echo what was said in the opening, but will have much more resonance for the reader who now has a deeper understanding of the research context.
%
%Restatement of the response: Similarly, the response can be restated in more meaningful detail for the reader who now has a better understanding of the problem.
%Roadmap: Brief indication of how the thesis will proceed.
